%BAB_3 LAPORAN KP

\chapter{PERANCANGAN SISTEM}

\section{Metode Perancangan sistem}
Perancangan robot serpent-2 ini diawali dengan menentukan metode yang tepat untuk mendesain dan membangun sistem secara keseluruhan meliputi perancangan elektronis,pemrograman pada Arduino Mega 2560, implementasi kinematika robot pada robot serpent-2, serta perancangan antarmuka pada \emph {processing ide}.Metode perancangan sistem meliputi diagram blok, flowchart cara kerja sistem, prinsip kerja dan perancangan tiap segmen-segmen yang dibutuhkan.
	
	\subsection{Diagram Blok Perancangan Sistem}
	Pada dasarnya, perancangan sistem untuk robot serphent-2 secara sederhana dapat dibagi menjadi tiga bagian. Ketiga perancangan ini merupakan hal yang sangat penting dan saling berkaitan.Perancangan robot serphent=2 jika digambarkan dalam diagram blok sistem dapat digambarkan seperti yang ditunjukkan dalam gambar 3.1
	\begin{figure}[H]
		\centering
		\includegraphics[width=\linewidth]{gambar/diagram_blok.jpg}
		\caption{Diagram metode perancangan sistem.}
	\end{figure}

\subsection{Flowchart Cara Kerja Sistem}
Kerja sistem, merupakan bagaimana robot serphent-2 melakukan tugasnya sesuai perintah yang dimasukkan dan kemudian dilaksanakan oleh aktuator.Robot serphent-2 memiliki kerja sistem yang tergolong ringkas yang mana didominasi oleh sistem maju tetapi juga memiliki sistem balik.  Kerja sistem dari robor serphent 2 jika dirancang dalam bentuk flowchat dapat ditunjukkan seperti dalam gambar 3.2

\begin{figure}[H]
	\centering
	\includegraphics[width=8cm	]{gambar/flowchart.png}
	\caption{Flowchat cara kerja sistem.}
\end{figure}
	
	\subsection{Perancangan Elektronis}
	Perancangan elektronis merupakan perancangan dasar pada pembuatan suatu sistem. Suatu sistem dapat bekerja secara maksimal karena terdiri dari komponen-komponen yang memiliki fungsi masing-masing. Komponen-komponen ini, disatukan kedalam sebuah \textit{Shield} \textit{Printed Circuit Board} (PCB). 
	
	\begin{enumerate}
		\item Pengendali motor DC yang digunakan adalah modul EMS 30A H-Bridge sebanyak tiga buah yang masing-masing untuk menggerakkan \textit{Shoulder, Elbow} dan perputaran \textit{Wrist}. Secara garis besar, fungsi modul pengendali motor ini adalah untuk mengendalikan arah dan kecepatan putaran motor DC sesuai instruksi kendali dari Arduino Mega 2560 pengguna.Modul akan menerima nilai yanf dikirimkan oleh Arduino Mega 2560 dan kemudian menggerakan motor servo yang sudah terhubung dengan \textit{shoulder, Elbow} dan perputaran dari \textit{Wirst}.
		
		\begin{figure}[H]
			\centering
			\includegraphics[width=5cm	]{gambar/driver_motor.jpg}
			\caption{Pengendali Motor DC EMS 30A H-Bridge}
		\end{figure}
		
		\item Potensiometer yang digunakan adalah jenis potensiometer \textit{rotary}. Potensiometer ini sebagai sensor posisi motor servo. Potensiometer terpasang pada setiap bagian motor servo sesuai dengan perputarannya dan akan memberikan keluaran berupa level tegangan yang berubah-ubah sesuai dengan posisi motor servo saat itu. Level tegangan tersebut kemudian dikirimkan kepada Arduino Mega 2560 sebagai sensor \textit{feedback} yang nantinya akan diproses untuk menyempurnakan posisi sesuai yang ditentukan.
		\begin{figure}[H]
			\centering
		%	\includegraphics[width=5cm	]{potensio.jpg}
			\caption{Mekanisme pemasangan potensiometer pada motor}
		\end{figure}
		
		\item Pengaturan pergerakan vertikal dari \textit{wirst} pada robot serphent-2 menggunakan sistem pneumatik silinder. Pada bagian buka tutup \textit{wirst} menggunakan masukan udara biasa untuk menutupnya dan membuang udara unutk membukanya. Udara tersebut didapat dari kompresor yang terhubung melalui selang dan dikontrol melalui sebuah relay yang bekerja pada tegangan 24v.
		
		\begin{figure}[H]
			\centering
		%	\includegraphics[width=5cm	]{relay.jpg}
			\caption{Relay pneumatik}
		\end{figure}
		\begin{figure}[H]
			\centering
		%	\includegraphics[width=5cm	]{relay2.jpg}
			\caption{Pneumatik Silinder}
		\end{figure}
		
		\item Relay yang bekerja pada tegangan 24v, pada Arduino Mega 2560 dikontrol melalui sinyal digital dengan bantuan rangkaian yang menggunakan TIP31A yang berfungsi untuk memutus atau membuka tegangan 24v. 
		\begin{figure}[H]
			\centering
		%	\includegraphics[width=5cm	]{relay3.jpg}
			\caption{Rangkaian skematik TIP31 sebagai \textit{switch}}
		\end{figure}
		
		\item Semua komponen-komponen yang dibutuhkan pada sistem kerja, disatukan ke dalam \textit{shield PCB} yang bertujuan agar meringkaskan serta memudahkan perangkaian elektronis. Rangkaian PCB dibuat melalui \textit{software} Eagle.
		\begin{figure}[H]
			\centering
		%	\includegraphics[width=15cm	]{skematik.pdf}
			\caption{Skematik rangkaian elektronis keseluruhan}
		\end{figure}
		
		
	\end{enumerate}



SCARA merupakan singkatan dari \emph{Selective Compliant Assembly Robot Arm}. Robot ini pertama kali dibuat oleh perusahaan USA bernama Adept pada 1984 dan diklasifikasikan sebagai robot industri. Sistem penggerak robot SCARA merupakan pergerakan langsung pada lengan tanpa bantuan sistem \emph{belt} keculai pada bagian \emph wirst, sehingga membuat mekanisme gerakannya bekerja cepat, sederhana namun tetap akurat. Robot ini banyak digunakan sebagai robot \emph {aseembly part} dengan ukuran yang kecil degan kecepatan sedang. 


Robot SCARA yang digunakan pada penelitian ini menggunakan robot SCARA dengan nama Serpent-2. Robot Serpent-2 memiliki dua \textit{horizontal joint} yaitu bagian \textit{shoulder, elbow}dan \textit{wrist} yang dikendalikan oleh motor servo. Sedangkan pada bagian \textit{vertical joint} yang berfungsi sebagai naik turun dan buka tutup dari \emph wirst, dikendalikan oleh pneumatik yang dikontrol oleh \emph {valve relay}. Sehingga, gerakan yang terdapat pada robot SCARA dapat diklasifikasikan sebagai gerakan mengambil dan menempatkan objek. 


\begin{table}[H]
	\centering
	\caption{Spesifikasi Robot Serpent-2}
	\resizebox{6cm}{!}{%
		\begin{tabular}{|l|l|}
			\hline
			Main arm length      & 360 mm$$\hspace{2cm} 		\\ \hline
			Fore arm length      & 290 mm$$  				\\ \hline
			Shoulder movement    & 180 °$$  		\\ \hline
			Elbow movement       & 200 °$$   		\\ \hline
			Wrist rotation       & 360 °$$ 		\\ \hline
			Up \& down movement  & 150 mm$$   				\\ \hline
			Maximum tip velocity & 3.0 kg$$  				\\ \hline
			
			\end{tabular}%
		}
		\end{table}
		
		Pada bagian motor servo, robot serpent-2 menggunakan tiga buah sensor \emph feedback yang berguna sebagai pemberi nilai posisi pada masing-masing motor servo. Sensor \emph feedback yang digunakan pada robot SCARA ini menggunakan potensiometer yang memberikan nilai analog dan kemudian diproses oleh Arduino Mega 2560. Nilai ini, nantinya untuk memproses gerak kinematika dari robot SCARA tersebut sesuai dengan posisi yang diinginkan.
		
		
		
		\section{Motor servo}
		Motor servo merupakan sebuah motor DC yang memiliki sistem \textit{feedback}. \textit{feedback} pada motor servo merupakan koreksi sudut motor DC terhadap sudut referensi \cite{Younkin2002}. pada robot Serpent-1 terdapat tiga buah motor DC, motor DC	 bagian wrist dan elbow merupakan motor DC yang identik.sehingga penulis hanya fokus membandingkan spesifikasi dua motor DC yaitu motor DC yang berada di bagian shoulder (\textit{main arm}) dan motor DC yang berada di bagian elbow (\textit{fore arm}). spesifikasi dua motor DC tersebut dapat dilihat pada tabel 2.2 sebagai berikut
		
		\begin{table}[H]
		\centering
		\caption{Spesifikasi Motor DC pada robot Serpent-1}
		\resizebox{11cm}{!}{%
			\begin{tabular}{|l|l|}
			\hline
			Moments of inertia of the main arm ($J_{1}$)    							& $0.0980kgm^{2}$ 				\\ \hline
			Moments of inertia of the fore arm ($J_{2}$)    							& $0.0115 kgm^{2}$ 				\\ \hline
			Masses of the main arm	($m_{1}$)											& $1.90kg$   					\\ \hline
			Masses of the fore arm  ($m_{2}$)     										& $0.93kg$   					\\ \hline
			Motor and equivalent inertias ($J_{m}$)      								& $3.3*10^{-6}kgm^{2}$ 			\\ \hline
			Back emf constants for main arm and fore arm motor ($K_{e1}=K_{e2}$)  		& $0.047Nm/A$   				\\ \hline
			Armature resistance for main arm and fore arm motor($R_{a1}=R_{a2}$)		& $3.5\Omega$  					\\ \hline
			Armatures inductances for main and fore arm motor  ($L_{a1}=L_{a2}$) 		& $1.3mH$ 						\\ \hline
			\end{tabular}%
		}
		\end{table}
		%%

	\subsection{Perancangan Elektronis}
	
	\subsection{Pemrograman Kontroller \& Implementasi PID}
	
	\subsection{Perancangan Antar Muka \& Inplementasi \textit{Invers Kinematic}}
	