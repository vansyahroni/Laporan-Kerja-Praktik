%!TEX root = ./template-skripsi.tex
%-------------------------------------------------------------------------------
% 								BAB I
% 							LATAR BELAKANG
%-------------------------------------------------------------------------------

\chapter{LATAR BELAKANG}

\section{Latar Belakang Masalah}
	Robotika merupakan salah satu bidang ilmu yang mempelajari desain, modeling, dan kendali robot. Saat ini, robot memiliki peranan yang sangat penting dalam membantu pekerjaan manusiacontohnya dalam bidang industri. Dalam bidang industri, robot umumnya terdapat pada proses manufaktur proses pengelasan, \textit{spray painting}, perakitan, \textit{milling} dan \textit{drilling} dapat dikerjakan dengan sebuah robot secara terus menerus dan berulang secara otomatis. Salah satu robot dalam proses manufaktur adalah Robot SCARA. Robot SCARA atau \emph{Selective Compilance Assembly Robot Arm} dalam dunia industri umumnya digigunakan dalam pemilahan barang.
	
Robot SCARA dapat bergerak secara optimal dan efisien karena sebuah persamaan kinematika. Persamaan kinematika yang diguanakan adalah \textit{inverse kinematic} dengan input berupa titik koordinat kartesius \textit{($x_{1}$,$y_{2}$)} dan output berupa nilai sudut untuk mengendalikan motor servo pada \textit{shoulder} dan \textit{elbow}.

Dalam membangun sebuah sistem kendali, dibutuhkan \textit{platfrom} antar muka sebagai jembatan antara \textit{user} dengan \textit{hardware}. Dalam penelitian ini program antar muka dirancang menggunakan \textit{software} Processing Ide. \emph Software ini memiliki beberapa keunggulan yang membuatnya lebih efektif dan cukup mudah untuk digunakan sebagai \textit{platform} antar muka. Keunggulan tersebut salah satunya ialah mudahnya sarana komunikasi terhadap \emph hardware yang digunakan. \emph Processing Ide juga dapat melakukan komunkasi dua arah yang berarti antara antar muka dan juga \emph hardware yang digunakan dapat saling berkirim data dan juga menerima data.
Oleh Karena itu, pada program kerja praktik ini dilakukan analisis robot SCARA berupa kinematika maju dan kinematika balik dengan perancangan antar muka berbasis \emph Processing Ide. Atas dasar tersebut penulis membuat judul kerja praktik berjudul "\textbf{Rancang Bangun Kendali Kinematika dan Antar Muka Robot SCARA Berbasis Processing Ide}" guna memberikan inovasi dan pengembangan pada sistem kendali kinematika pada robot SCARA yang dapat diimplementasikan sebagai sarana belajar sistem kendali di Laboratorium Instrumentasi dan Kendali Sekolah Vokasi Universitas Gadjah Mada.\\

\section{Tujuan}
Tujuan penulis melakukan penelitian dibagi menjadi dua, yaitu tujuan secara umum dan tujuan secara khusus.

\subsection{Tujuan Umum}

Memenuhi salah satu syarat kelulusan program kerja praktik program studi Teknologi Listrik Sekolah Vokasi Universitas Gadjah Mada.

\subsection{Tujuan Khusus}

\begin{enumerate}
	\item Merancang sistem kinematika robot SCARA, baik kinematika maju maupun balik,
	\item Merancang antarmuka penggunaan robot SCARA dengan \emph {software Processing Ide.}
\end{enumerate}


\section{Metode Kerja Praktik}
Pengerjaan dan penyusunan Kerja Praktik ini menggunakan beberapa metode.

\begin{enumerate}
	\item Metode pustaka\\
	Metode dengan memahami dan mempelajari berbagai jurnal yang berkaitan dengan robot SCARA, gerak kinematika, dan \emph {Processing Ide}.
	\item Metode perancangan alat\\
	Metode dengan melakukan pembuatan desain elektronis dan mekanis, pemrograman, serta perancangan antarmuka menggunakan \emph{ Processing Ide].
	\item Metode pengujian\\
	Metode dengan melakukan pengujian terhadap paramater-parameter yang dibutuhkan seperti pergerakan dari motor servo, pembacaan nilai umpan balik dari motor oleh potensiometer, serta pengujian terhadap kinematika robot SCARA yang tertampil pada antarmuka \emph {Processing Ide.}

\end{enumerate}

\section{Sistematika Penulisan}
Penulisan laporan Kerja praktik ini dilakukan dengan mengikuti sistematika sebagai berikut:\\
\noindent
\textbf{BAB I\hspace*{0.6cm}: PENDAHULUAN}\\
\noindent
Memuat latar belakang masalah, maksud dan tujuan, batasan masalah, metodologi penetilian, dan sistematika penulisan.\\
\noindent
\textbf{BAB II\hspace*{0.5cm}: LANDASAN TEORI}\\
\noindent
Memuat gambaran umum robot SCARA, kinematika maju dan kinematika balik, dan \emph{Processing Ide.}\\
\textbf{BAB III\hspace*{0.375cm}:  PERANCANGAN SISTEM}\\
\noindent
Memuat perancangan sistem secara umum, dimulai dari perancangan elektronis, mekanis, pemrograman hingga pada antarmuka \emph {processing ide.}\\
\textbf{BAB IV\hspace*{0.4cm}: PENGUJIAN SISTEM}\\
\noindent
Memuat pengujian dan anilisa terhadap hasil yang didapat seperti analisa kerja dari sistem penggerak, pembacaan sensor, serta pengujian sistem secara keseluruhan dengan \emph {processing ide.}
\\
\textbf{BAB V\hspace*{0.6cm}: PENUTUP}
Memuat kesimpulan dari perancangan hingga pada hasil pengujian, dan berisi saran-saran untuk pengembangan lebih lanjut.
\\



% Baris ini digunakan untuk membantu dalam melakukan sitasi
% Karena diapit dengan comment, maka baris ini akan diabaikan
% oleh compiler LaTeX.
\begin{comment}
\bibliography{daftar-pustaka}
\end{comment}
