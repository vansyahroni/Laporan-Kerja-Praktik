%BAB_2 LAPORAN KP
\chapter{LANDASAN TEORI}

\section{Gambaran Umum Robot Lengan}

Robot adalah adalah sebuah alat yang terdiri dari gabungan mekanik dan elektronik yang dapat melakukan tugas fisik, baik menggunakan pengawasan dan kendali manusia maupun secara otomatis. Robot dapat melakukan suatu tugas secara berulang tanpa merasa lelah sehingga robot banyak digunakan dalam dunia industri khususnya pada bidang produksi. Salah satu jenis robot yang sering dalam bidang produksi adalah sistem lengan robot.

Robot lengan adalah robot yang memiliki bentuk fisik seperti halnya lengan pada manusia dan memiliki derajat kebebasan (degre of freedom) tertentu bergantung pada jumlah sendi yang digunakan. Dengan begitu robot lengan terdiri dari beberapa jenis. Robot lengan pada bidang industri biasa digunakan sebagai actuator untuk mengambil dan meletakkan suatu objek secara terus menerus.
	

Pada umumnya struktur robot lengan terdiri dari beberapa bagian.  Bagian utama adalah struktur mekanik (manipulator) yang merupakan susunan kerangka yang tidak dapat digerakkan (rigid) dan lengan (link) yang satu sama lain terhubung oleh sendi (joint). Dengan adanya joint yang menghubungkan dua link menjadi satu kesatuan sehingga joint membentuk satu derajat kebebasan. Jika diibaratkan dengan tubuh manusia, link adalah tulang sedangkan joint adalah sendi-sendinya. Joint memiliki dua pergerakan, yaitu pergerakan revolute joint (gerak berputar) dan prismatic joint (gerak bergeser) seperti yang ditunjukkan oleh Gambar 2.1

	\begin{figure}[H]
	\centering
	\includegraphics[width=10cm]{gambar/joint.png}
	\caption{Jenis-Jenis \emph Joint}
\end{figure}

Pada ujung pangkal lengan, robot lengan umumnya menggunakan gripper yang dapat dipakai untuk memindahkan suatu objek. Robot lengan dalam menjalankan tugasnya dikontrol menggunakan sensor serta aktuator yang telah dirancang untuk melakukan tugas sesuai dari yang diperintahkan. Perpaduan antara sensor dan aktuator ini yang menyebabkan robot lengan dapat bekerja secara optimal dan presisi.

\subsection{\emph{Degress of Freedom }}
Degress of freedom (DOF) merupakan sebuah konfigurasi yang dapat meminimalkan spesifikasi dengan menggunakan n parameter yang dapat menyatakan posisi suatu system pada setiap saat. Biasanya, robot lengan mempunyai paling sedikit enam independen derajat kebebasan: tiga derajat kebebasan untuk translasi dan tiga derajat kebebasan untuk rotasi. Umumnya untuk robot lengan paling tidak memiliki tiga derajat kebebasan untuk dapat memiliki workspace yang cukup. Workspace dari sebuah robot lengan merupakan total volume yang dapat dijangkau oleh end effector dari pergerakan semua jointnya dari titik minimum hingga maksimum. 

\subsection{Konfigurasi Robot Lengan}
Pada dasarnya, berbagai jenis dari robot lengan dapat dibedakan dari konfigurasinya. konfigurasi robot lengan merupakan perpaduan antara pergerakan joint yang dimiliki oleh robot lengan. konfigurasi ini memiliki tipe yang berbeda-beda sehingga \emph workspace yang dimiliki pada tiap robot lengan pasti berbeda.

\subsubsection{A. Konfigurasi Articulated (Revolute - Revolute - Revolute)} 
Articulated manipulator ini pada dasarnya mempunyai jenis revolute joint pada ketiga joint robot lengan (\emph {wrist, shoulder, elbow}). Dengan konfigurasi ini, robot lengan dengan konfigurasi Articulated dapat memiliki variasi DOF yang banyak. DOF yang daoat dihasilkan dengan robot lengan dengan konfigurasi seperti ini adalah tiga DOF hingga sampai dengan enam DOF tergantung dari kebutuhan dan fungsi yang akan dilakukan oleh robot lengan. Konfigurasi dari joint revolute ini menjadikan robot lengan jenis ini mempunyai kebebasan yang besar dari pergerakannya dalam ruang yang kecil sehingga menjadikan jenis konfigurasi articulated manipulator ini banyak dipakai dan memiki desain yang populer. Konfigurasi Articulated ini dapat dianalisakan seperti yang ada pada Gambar 2. 2.

\subsubsection{B. Konfigurasi Spherical (Revolute – Revolute – Prismatic)   )} 

Konfigurasi spherical merupakan konfigurasi yang mempunyai dua buah joint revolute dan satu buah joint prismatic. Joint prismatic berada ini joint ketiga atau pada bagian elbow. Sementara dua joint lainnya berada di shoulder dan waist. Sruktur dari konfigurasi Spherical seperti pada Gambar 2.3

\subsubsection{C. Konfigurasi SCARA (Revolute – Revolute – Prismatic) } 

Konfigurasi Selective Compliant Articulated Robot for Assembly (SCARA) merupakan konfigurasi yang mempunyai dua buah joint revolute dan satu buah joint prismatic sama seperti konfigurasi Spherical. Meskipun SCARA memiliki struktur joint revolute – revolute – prismatic (RRP) sama seperti konfigurasi yang dimiliki spherical, struktur ini sedikit berbeda dengan konfigurasi spherical dari tampilannya maupun dari jarak workspace nya. Tidak seperti konfigurasi spherical, dimana z0 tegak lurus terhadap 1, dan z1 tegak lurus dengan z2, konfigurasi SCARA memiliki struktur z0, z1, dan z2 yang paralel. Struktur dari konfigurasi SCARA seperti yang ditunjukkan Gambar 2.4

\subsubsection{D. Konfigurasi Cylindrical (Revolute – Prismatic – Prismatic) } 

Konfigurasi Cylindrical merupakan konfigurasi yang mempunyai satu buah joint revolute dan dua buah joint prismatic. Joint revolute menghasilkan pergerakan rotasi di base/ waist, sementara joint prismatic berada di bagian shoulder dan elbow. Struktur dari konfigurasi Cylindrical seperti yang ditunjukkan oleh Gambar 2.5

\subsubsection{E. Konfigurasi Cartesian (Prismatic – Prismatic – Prismatic)  } 

Konfigurasi cartesian mempunyai tiga buah joint prismatic. Variabel joint dari konfigurasi prismatic adalah koordinat cartesian dari end-effector dengan memperhatikan letak base dari robot lengan. Seperti yang diperkirakan kinematika dari jenis konfigurasi ini adalah yang paling sederhana dari semua konfigurasi robot lengan. Konfigurasi cartesian sangat berguna untuk penyusunan suatu barang di bidang datar seperti mesin laser, kargo atau memindahkan barang. Struktur dari konfigurasi Cartesian ditunjukkan pada Gambar 2.6


SCARA merupakan singkatan dari \emph{Selective Compliant Assembly Robot Arm}. Robot ini pertama kali dibuat oleh perusahaan USA bernama Adept pada 1984 dan diklasifikasikan sebagai robot industri. Sistem penggerak robot SCARA merupakan pergerakan langsung pada lengan tanpa bantuan sistem \emph{belt} keculai pada bagian \emph wirst, sehingga membuat mekanisme gerakannya bekerja cepat, sederhana namun tetap akurat. Robot ini banyak digunakan sebagai robot \emph {aseembly part} dengan ukuran yang kecil degan kecepatan sedang. 

Robot SCARA yang digunakan pada penelitian ini menggunakan robot SCARA dengan nama Serpent-2. Robot Serpent-2 memiliki dua \textit{horizontal joint} yaitu bagian \textit{shoulder, elbow}dan \textit{wrist} yang dikendalikan oleh motor servo. Sedangkan pada bagian \textit{vertical joint} yang berfungsi sebagai naik turun dan buka tutup dari \emph wirst, dikendalikan oleh pneumatik yang dikontrol oleh \emph {valve relay}. Sehingga, gerakan yang terdapat pada robot SCARA dapat diklasifikasikan sebagai gerakan mengambil dan menempatkan objek. 
\begin{table}[H]
	\centering
	\caption{Spesifikasi Robot Serpent-2}
	\resizebox{6cm}{!}{%
		\begin{tabular}{|l|l|}
			\hline
			Main arm length      & 360 mm$$\hspace{2cm} 		\\ \hline
			Fore arm length      & 290 mm$$  				\\ \hline
			Shoulder movement    & 180 °$$  		\\ \hline
			Elbow movement       & 200 °$$   		\\ \hline
			Wrist rotation       & 360 °$$ 		\\ \hline
			Up \& down movement  & 150 mm$$   				\\ \hline
			Maximum tip velocity & 3.0 kg$$  				\\ \hline
			
			\end{tabular}%
		}
		\end{table}
		
		Pada bagian motor servo, robot serpent-2 menggunakan tiga buah sensor \emph feedback yang berguna sebagai pemberi nilai posisi pada masing-masing motor servo. Sensor \emph feedback yang digunakan pada robot SCARA ini menggunakan potensiometer yang memberikan nilai analog dan kemudian diproses oleh Arduino Mega 2560. Nilai ini, nantinya untuk memproses gerak kinematika dari robot SCARA tersebut sesuai dengan posisi yang diinginkan.



\section{Motor servo}
	Motor servo merupakan sebuah motor DC yang memiliki sistem \textit{feedback}. \textit{feedback} pada motor servo merupakan koreksi sudut motor DC terhadap sudut referensi \cite{Younkin2002}. pada robot Serpent-1 terdapat tiga buah motor DC, motor DC	 bagian wrist dan elbow merupakan motor DC yang identik.sehingga penulis hanya fokus membandingkan spesifikasi dua motor DC yaitu motor DC yang berada di bagian shoulder (\textit{main arm}) dan motor DC yang berada di bagian elbow (\textit{fore arm}). spesifikasi dua motor DC tersebut dapat dilihat pada tabel 2.2 sebagai berikut

\begin{table}[H]
	\centering
	\caption{Spesifikasi Motor DC pada robot Serpent-1}
	\resizebox{11cm}{!}{%
		\begin{tabular}{|l|l|}
			\hline
			Moments of inertia of the main arm ($J_{1}$)    							& $0.0980kgm^{2}$ 				\\ \hline
			Moments of inertia of the fore arm ($J_{2}$)    							& $0.0115 kgm^{2}$ 				\\ \hline
			Masses of the main arm	($m_{1}$)											& $1.90kg$   					\\ \hline
			Masses of the fore arm  ($m_{2}$)     										& $0.93kg$   					\\ \hline
			Motor and equivalent inertias ($J_{m}$)      								& $3.3*10^{-6}kgm^{2}$ 			\\ \hline
			Back emf constants for main arm and fore arm motor ($K_{e1}=K_{e2}$)  		& $0.047Nm/A$   				\\ \hline
			Armature resistance for main arm and fore arm motor($R_{a1}=R_{a2}$)		& $3.5\Omega$  					\\ \hline
			Armatures inductances for main and fore arm motor  ($L_{a1}=L_{a2}$) 		& $1.3mH$ 						\\ \hline
		\end{tabular}%
	}
\end{table}

\section{Kinematika Robot}
	Pada bagian motor servo, robot serpent-2 menggunakan tiga buah sensor \emph feedback yang berguna sebagai pemberi nilai posisi pada masing-masing motor servo. Sensor \emph feedback yang digunakan pada robot SCARA ini menggunakan potensiometer yang memberikan nilai analog dan kemudian diproses oleh Arduino Mega 2560. Nilai ini, nantinya untuk memproses gerak kinematika dari robot SCARA tersebut sesuai dengan posisi yang diinginkan.		
	
\section{Processing IDE}
	\begin{figure}[H]
	\centering
	\includegraphics[width=4.33cm]{gambar/logo_processing.png}
	\caption{Processing IDE}
	\end{figure}
	\emph{Processing} adalah lingkungan pemrograman sederhana yang dibuat untuk  memudahkan pengembangkan aplikasi yang berorientasi visual dengan penekanan pada animasi dan menyediakan respon balik yang instan kepada pengguna melalui interaksi didalamnnya. Para pengembang menginginkan cara untuk "membuat sketsa" ide dalam kode. Karena kemampuannya telah berkembang selama dekade terakhir, \emph{Processing} telah digunakan untuk pekerjaan tingkat produksi yang lebih maju. Awalnya dibangun sebagai ekstensi khusus domain ke Java yang ditargetkan untuk seniman dan desainer, Processing telah berevolusi menjadi desain penuh dan alat \emph{prototyping} yang digunakan untuk pekerjaan instalasi skala besar, gambar gerak, dan visualisasi data yang kompleks.
