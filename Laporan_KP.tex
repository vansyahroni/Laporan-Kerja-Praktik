	\documentclass{DTEDI_KP}

\usepackage[titles]{tocloft}
\renewcommand\cftfigpresnum{Gambar\ }
\renewcommand\cfttabpresnum{Tabel\  }

\usepackage{hyperref}
\newlength{\mylenf}
\settowidth{\mylenf}{\cftfigpresnum}
\setlength{\cftfignumwidth}{\dimexpr\mylenf+2em}
\setlength{\cfttabnumwidth}{\dimexpr\mylenf+2em}

\usepackage[labelfont=bf]{caption}

\usepackage{caption}
\usepackage{subcaption}
\usepackage{graphicx}
\usepackage{float}
\usepackage{textcomp}
\usepackage{amsmath}

\titleind{KINEMATIKA DAN ANTARMUKA ROBOT SCARA BERBASIS PROCESSSING IDE}

\fullname {IVAN SYAHRONI HERMAWAN}

\idnum {17/415746/SV/13611}

\approvaldate {2 Agustus 2019}

\degree {Teknologi Listrik}

\yearsubmit {2019}

\program {Teknologi Listrik}

\dept {Departemen Teknik Elektro dan Informatika}

\secondsupervisor {Fahmizal,S.T.,M.Sc.}

\secondnip {111198807201609101}

\firstsupervisor {Ma’un Budiyanto,  S.T., M.T.}

\firstnip {197007071999031002}

\begin{document}
	
	\cover
	
	\approvalpage
	
	\preface
	
Puji syukur penulis panjatkan kehadirat Tuhan Yang Maha Esa karena hanya dengan rahmat dan hidayah-Nya, laporan kerja praktik ini dapat diselesaikan tanpa halangan yang berarti. Keberhasilan dalam menyusun laporan kerja praktik ini tidak lepas dari bantuan berbagai pihak yang mana dengan tulus dan ikhlas membe- rikan masukan guna sempurnanya laporan kerja praktik ini.  Oleh karena itu dalam kesempatan ini, dengan kerendahan hati penulis mengucap terimakasih kepada:
	
	\begin{enumerate}
		\item Bapak Ma’un Budiyanto, S.T., M.T. selaku Ketua Program Studi Teknologi Listrik Universitas Gadjah Mada,
		\item  Bapak Fahmizal, S.T., M.Sc selaku dosen pembimbing pertama yang telah memberikan banyak bantuan, bimbingan, serta arahan dalam kerja praktik,
		\item Seluruh Dosen di Teknologi Listrik Sekolah Vokasi Universitas Gadjah Mada, yang tidak bisa disebutkan satu-satu, atas ilmu dan bimbingannya,
		\item Ibu dan Bapak yang selama ini telah sabar membimbing, mengarahkan, dan mendoakan penulis tanpa kenal lelah untuk selama-lamanya, dan
	\end{enumerate}

Penulis menyadari bahwa penyusunan laporan kerja praktik ini jauh dari sempurna. Kritik dan saran dapat ditujukan langsung pada e-mail saya. Akhir kata penulis mo- hon maaf yang sebesar-besarnya apabila terdapat kekeliruan di dalam penulisan kerja praktik ini.

\vspace{0.1cm}

Wassalamu’alaikum Wr. Wb.

	\begin{tabular}{p{7.5cm}c}
	&Yogyakarta, 2 Agustus 2019\\
	&\\
	&\\
	&\\
	&\\
	&\textbf{Penulis}
	\end{tabular}

\tableofcontents
\addcontentsline{toc}{chapter}{DAFTAR ISI}
\listoftables
\addcontentsline{toc}{chapter}{DAFTAR TABEL}
\listoffigures
\addcontentsline{toc}{chapter}{DAFTAR GAMBAR}

\begin{abstractind}
Penelitian ini bertujuan untuk melakukan pengoperasian terhadap robot SCA- RA. SCARA merupakan akronim untuk Selective Compliance Assembly Robot Arm dimana robot ini dapat bergerak dalam dua aksis, yaitu horisontal dan vertikal. Per- gerakan Robot ini menggunakan kedua lengan untuk pergerakan horisontal dan satu lengan untuk pererakan vertikal. Robot SCARA ini dioperasikan dengan bantuan an- tar muka yang dibuat dari Processing yang dibuat menggunakan program bahasa c. Antarmuka yang ditampilkan menunjukan pengoperasian robot SCARA mulai dari kinematika maju dan kinematika balik.

Kata kunci : SCARA, Processing, Kendali, inverse kinematic, Forward Kinematics.
\end{abstractind}

\begin{abstracteng}

\end{abstracteng}

\newpage
\setcounter{page}{1}
\pagenumbering{arabic}

%BAB-1 Laporan KP

\chapter{PENDAHULUAN}

\section{Latar Belakang Masalah}

	Perkembangan teknologi serta ilmu pengetahuan pada masa ke masa semakin berkembang. Perkembangan ini berjalan seiring dengan penelitian-penelitian di berbagai disiplin ilmu khususnya dalam bidang instrumentasi dan kendali. Hal ini dapat dilihat dari banyaknya penggunaan sistem instrumentasi dan kendali dalam dunia industri seperti pengguanaan robot dalam menyelesaiakan pekerjaan manusia.Untuk itu perancangan robot merupakan salah satu solusi untuk memenuhi tuntutan dalam membantu kebutuhan manusia.
	
	Pemilihan robot untuk menggantikan pekerjaan manusia tidak terlepas dengan berbagai kelebihannya. Robot dapat melakukan suatu pekerjaan yang sama dan berulang tanpa merasakan lelah seperti halnya manusia. Pekerjaan ini lah yang biasa ditemukan dalam bidang industri khususnya pada bagian produksi. Robot dengan sistem lengan robot (\emph robot arm sistem) merupakan salah satu jenis robot yang dominan berada dalam bidang industri. 
	
	Robot lengan memiliki berbagai jenis, salah satunya adalah robot SCARA (\emph{Selective Compilance Assembly Robot Arm}). Robot SCARA dapat bergerak secara optimal dan efisien karena sebuah persamaan kinematika. Persamaan kinematika yang diguanakan adalah \textit{inverse kinematic} dengan input berupa titik koordinat kartesius \textit{($x_{1}$,$y_{2}$)} dan output berupa nilai sudut untuk mengendalikan motor servo pada \textit{shoulder} dan \textit{elbow}.
	
	Dalam membangun sebuah sistem kendali, dibutuhkan \textit{platfrom} antar muka sebagai jembatan antara \textit{user} dengan \textit{hardware}. Dalam penelitian ini program antar muka dirancang menggunakan \textit{software} Processing Ide. \emph Software ini memiliki beberapa keunggulan yang membuatnya lebih efektif dan cukup mudah untuk digunakan sebagai \textit{platform} antar muka. Keunggulan tersebut salah satunya ialah mudahnya sarana komunikasi terhadap \emph hardware yang digunakan. \emph Processing Ide juga dapat melakukan komunkasi dua arah yang berarti antara antar muka dan juga \emph hardware yang digunakan dapat saling berkirim data dan juga menerima data.
	Oleh Karena itu, pada program kerja praktik ini dilakukan analisis robot SCARA berupa kinematika maju dan kinematika balik dengan perancangan antar muka berbasis \emph Processing Ide. Atas dasar tersebut penulis membuat judul kerja praktik berjudul "\textbf{Rancang Bangun Kendali Kinematika dan Antar Muka Robot SCARA Berbasis Processing Ide}" guna memberikan inovasi dan pengembangan pada sistem kendali kinematika pada robot SCARA yang dapat diimplementasikan sebagai sarana belajar sistem kendali di Laboratorium Instrumentasi dan Kendali Sekolah Vokasi Universitas Gadjah Mada.\\


\section{Tujuan Penelitian}
adapun tujuan dalam melaksanakan penelitian terbagi menjadi dua, yaitu tujuan secara umum dan secara khusus sebagai berikut

	\subsection{Tujuan Umum}
	
		Untuk memenuhi salah satu syarat kelulusan program kerja praktik program studi Teknologi Instrumentasi Sekolah Vokasi Universitas Gadjah Mada
		
	\subsection{Tujuan Khusus}
	
	\begin{enumerate}
		\item Merancang antar muka untuk robot Serpent-1 berbasis LabVIEW
		\item Mengimplementasikan persamaan kinematika Denavit-Hartenberg pada kinematika robot Serpent-1
	\end{enumerate}
	
\section{Batasan Penelitian}
	Pembatasan masalah diperlukan untuk mempermudah pelaksanaan penulisan laporan kerja praktik sehingga tidak menyimpang dari judul laporan. Lingkup pembatasan masalah dalam Laporan kerja praktik ini dibatasi pada :
	
	\begin{enumerate}
		
		\item Komunikasi antara LabVIEW dengan Arduino Menggunakan NI-VISA
		\item Analisis model \textit{elbow planar manipulator} Denavit-Hartenberg hanya dilakukan pada \textit{Horizontal joint}
		
	\end{enumerate}

\section{Metode Kerja Praktik}
metode yang digunakan dalam pengerjaan proyek dan penyusunan laporan kerja praktik ini dilakukan dengan metode sebagai berikut

\begin{enumerate}
	\item Metode pustaka\\
	Metode ini dilakukan dengan membaca dan mempelajari jurnal-jurnal yang memiliki topik seputar SCARA,LabVIEW, dan \textit{invers kinematic}
	
	\item Metode perancangan alat\\
	Metode ini dilakukan dengan membuat desain elektronis, perancangan kinematika robot, serta perancangan software antar muka robot.
	
	\item Metode pengujian\\
	Metode ini dilakukan dengan cara melakukan pengujian terhadap kinerja sensor, kinerja aktuator, dan kinerja sensor dan aktuator secara bersamaan.
	
\end{enumerate}

\section{Sistematika Penulisan}
Penulisan laporan kerja praktik ini dilakukan dengan mengikuti sistematika sebagai berikut :\\
\noindent
\textbf{BAB I\hspace*{0.6cm}: PENDAHULUAN}\\
\noindent
Memuat latar belakang masalah,tujuan, dan maksud,batasan masalah, metodologi penetilian, dan sistematika penulisan.\\
\noindent
\textbf{BAB II\hspace*{0.5cm}: LANDASAN TEORI}\\
\noindent
Memuat gambaran umum SCARA, \textit{invers kinematic} dengan analisis Denavit-Hartenberg, Software LabVIEW,dan Arduino Mega 2560.\\
\textbf{BAB III\hspace*{0.375cm}:  PERANCANGAN SISTEM}\\
\noindent
Memuat perancangan sistem secara umum, meliputi perancangan dari segi elektronis dan software dari robot Serpent-1.\\
\textbf{BAB IV\hspace*{0.4cm}: PENGUJIAN SISTEM}\\
\noindent
Memuat pengujian dan analisis kerja dari sistem aktuator dan sensor dari Serpent, serta pengujian sistem secara keseluruhan.\\
\textbf{BAB V\hspace*{0.6cm}: PENUTUP}\\
Memuat kesimpulan dari perancangan, pembuatan, pengujian, dan analisis kerja robot Serpent-1, serta berisi saran-saran untuk pengembangan lebih lanjut.\\
%BAB_2 LAPORAN KP
\chapter{LANDASAN TEORI}

\section{Gambaran Umum Robot Lengan}

Robot adalah adalah sebuah alat yang terdiri dari gabungan mekanik dan elektronik yang dapat melakukan tugas fisik, baik menggunakan pengawasan dan kendali manusia maupun secara otomatis. Robot dapat melakukan suatu tugas secara berulang tanpa merasa lelah sehingga robot banyak digunakan dalam dunia industri khususnya pada bidang produksi. Salah satu jenis robot yang sering dalam bidang produksi adalah sistem lengan robot.

Robot lengan adalah robot yang memiliki bentuk fisik seperti halnya lengan pada manusia dan memiliki derajat kebebasan (degre of freedom) tertentu bergantung pada jumlah sendi yang digunakan. Dengan begitu robot lengan terdiri dari beberapa jenis. Robot lengan pada bidang industri biasa digunakan sebagai actuator untuk mengambil dan meletakkan suatu objek secara terus menerus.
	

Pada umumnya struktur robot lengan terdiri dari beberapa bagian.  Bagian utama adalah struktur mekanik (manipulator) yang merupakan susunan kerangka yang tidak dapat digerakkan (rigid) dan lengan (link) yang satu sama lain terhubung oleh sendi (joint). Dengan adanya joint yang menghubungkan dua link menjadi satu kesatuan sehingga joint membentuk satu derajat kebebasan. Jika diibaratkan dengan tubuh manusia, link adalah tulang sedangkan joint adalah sendi-sendinya. Joint memiliki dua pergerakan, yaitu pergerakan revolute joint (gerak berputar) dan prismatic joint (gerak bergeser) seperti yang ditunjukkan oleh Gambar 2.1

	\begin{figure}[H]
	\centering
	\includegraphics[width=10cm]{gambar/joint.png}
	\caption{Jenis-Jenis \emph Joint}
\end{figure}

Pada ujung pangkal lengan, robot lengan umumnya menggunakan gripper yang dapat dipakai untuk memindahkan suatu objek. Robot lengan dalam menjalankan tugasnya dikontrol menggunakan sensor serta aktuator yang telah dirancang untuk melakukan tugas sesuai dari yang diperintahkan. Perpaduan antara sensor dan aktuator ini yang menyebabkan robot lengan dapat bekerja secara optimal dan presisi.

\subsection{\emph{Degress of Freedom }}
Degress of freedom (DOF) merupakan sebuah konfigurasi yang dapat meminimalkan spesifikasi dengan menggunakan n parameter yang dapat menyatakan posisi suatu system pada setiap saat. Biasanya, robot lengan mempunyai paling sedikit enam independen derajat kebebasan: tiga derajat kebebasan untuk translasi dan tiga derajat kebebasan untuk rotasi. Umumnya untuk robot lengan paling tidak memiliki tiga derajat kebebasan untuk dapat memiliki workspace yang cukup. Workspace dari sebuah robot lengan merupakan total volume yang dapat dijangkau oleh end effector dari pergerakan semua jointnya dari titik minimum hingga maksimum. 

\subsection{Konfigurasi Robot Lengan}
Pada dasarnya, berbagai jenis dari robot lengan dapat dibedakan dari konfigurasinya. konfigurasi robot lengan merupakan perpaduan antara pergerakan joint yang dimiliki oleh robot lengan. konfigurasi ini memiliki tipe yang berbeda-beda sehingga \emph workspace yang dimiliki pada tiap robot lengan pasti berbeda.

\subsubsection{A. Konfigurasi Articulated (Revolute - Revolute - Revolute)} 
Articulated manipulator ini pada dasarnya mempunyai jenis revolute joint pada ketiga joint robot lengan (\emph {wrist, shoulder, elbow}). Dengan konfigurasi ini, robot lengan dengan konfigurasi Articulated dapat memiliki variasi DOF yang banyak. DOF yang daoat dihasilkan dengan robot lengan dengan konfigurasi seperti ini adalah tiga DOF hingga sampai dengan enam DOF tergantung dari kebutuhan dan fungsi yang akan dilakukan oleh robot lengan. Konfigurasi dari joint revolute ini menjadikan robot lengan jenis ini mempunyai kebebasan yang besar dari pergerakannya dalam ruang yang kecil sehingga menjadikan jenis konfigurasi articulated manipulator ini banyak dipakai dan memiki desain yang populer. Konfigurasi Articulated ini dapat dianalisakan seperti yang ada pada Gambar 2. 2.

\subsubsection{B. Konfigurasi Spherical (Revolute – Revolute – Prismatic)   )} 

Konfigurasi spherical merupakan konfigurasi yang mempunyai dua buah joint revolute dan satu buah joint prismatic. Joint prismatic berada ini joint ketiga atau pada bagian elbow. Sementara dua joint lainnya berada di shoulder dan waist. Sruktur dari konfigurasi Spherical seperti pada Gambar 2.3

\subsubsection{C. Konfigurasi SCARA (Revolute – Revolute – Prismatic) } 

Konfigurasi Selective Compliant Articulated Robot for Assembly (SCARA) merupakan konfigurasi yang mempunyai dua buah joint revolute dan satu buah joint prismatic sama seperti konfigurasi Spherical. Meskipun SCARA memiliki struktur joint revolute – revolute – prismatic (RRP) sama seperti konfigurasi yang dimiliki spherical, struktur ini sedikit berbeda dengan konfigurasi spherical dari tampilannya maupun dari jarak workspace nya. Tidak seperti konfigurasi spherical, dimana z0 tegak lurus terhadap 1, dan z1 tegak lurus dengan z2, konfigurasi SCARA memiliki struktur z0, z1, dan z2 yang paralel. Struktur dari konfigurasi SCARA seperti yang ditunjukkan Gambar 2.4

\subsubsection{D. Konfigurasi Cylindrical (Revolute – Prismatic – Prismatic) } 

Konfigurasi Cylindrical merupakan konfigurasi yang mempunyai satu buah joint revolute dan dua buah joint prismatic. Joint revolute menghasilkan pergerakan rotasi di base/ waist, sementara joint prismatic berada di bagian shoulder dan elbow. Struktur dari konfigurasi Cylindrical seperti yang ditunjukkan oleh Gambar 2.5

\subsubsection{E. Konfigurasi Cartesian (Prismatic – Prismatic – Prismatic)  } 

Konfigurasi cartesian mempunyai tiga buah joint prismatic. Variabel joint dari konfigurasi prismatic adalah koordinat cartesian dari end-effector dengan memperhatikan letak base dari robot lengan. Seperti yang diperkirakan kinematika dari jenis konfigurasi ini adalah yang paling sederhana dari semua konfigurasi robot lengan. Konfigurasi cartesian sangat berguna untuk penyusunan suatu barang di bidang datar seperti mesin laser, kargo atau memindahkan barang. Struktur dari konfigurasi Cartesian ditunjukkan pada Gambar 2.6


SCARA merupakan singkatan dari \emph{Selective Compliant Assembly Robot Arm}. Robot ini pertama kali dibuat oleh perusahaan USA bernama Adept pada 1984 dan diklasifikasikan sebagai robot industri. Sistem penggerak robot SCARA merupakan pergerakan langsung pada lengan tanpa bantuan sistem \emph{belt} keculai pada bagian \emph wirst, sehingga membuat mekanisme gerakannya bekerja cepat, sederhana namun tetap akurat. Robot ini banyak digunakan sebagai robot \emph {aseembly part} dengan ukuran yang kecil degan kecepatan sedang. 

Robot SCARA yang digunakan pada penelitian ini menggunakan robot SCARA dengan nama Serpent-2. Robot Serpent-2 memiliki dua \textit{horizontal joint} yaitu bagian \textit{shoulder, elbow}dan \textit{wrist} yang dikendalikan oleh motor servo. Sedangkan pada bagian \textit{vertical joint} yang berfungsi sebagai naik turun dan buka tutup dari \emph wirst, dikendalikan oleh pneumatik yang dikontrol oleh \emph {valve relay}. Sehingga, gerakan yang terdapat pada robot SCARA dapat diklasifikasikan sebagai gerakan mengambil dan menempatkan objek. 
\begin{table}[H]
	\centering
	\caption{Spesifikasi Robot Serpent-2}
	\resizebox{6cm}{!}{%
		\begin{tabular}{|l|l|}
			\hline
			Main arm length      & 360 mm$$\hspace{2cm} 		\\ \hline
			Fore arm length      & 290 mm$$  				\\ \hline
			Shoulder movement    & 180 °$$  		\\ \hline
			Elbow movement       & 200 °$$   		\\ \hline
			Wrist rotation       & 360 °$$ 		\\ \hline
			Up \& down movement  & 150 mm$$   				\\ \hline
			Maximum tip velocity & 3.0 kg$$  				\\ \hline
			
			\end{tabular}%
		}
		\end{table}
		
		Pada bagian motor servo, robot serpent-2 menggunakan tiga buah sensor \emph feedback yang berguna sebagai pemberi nilai posisi pada masing-masing motor servo. Sensor \emph feedback yang digunakan pada robot SCARA ini menggunakan potensiometer yang memberikan nilai analog dan kemudian diproses oleh Arduino Mega 2560. Nilai ini, nantinya untuk memproses gerak kinematika dari robot SCARA tersebut sesuai dengan posisi yang diinginkan.



\section{Motor servo}
	Motor servo merupakan sebuah motor DC yang memiliki sistem \textit{feedback}. \textit{feedback} pada motor servo merupakan koreksi sudut motor DC terhadap sudut referensi \cite{Younkin2002}. pada robot Serpent-1 terdapat tiga buah motor DC, motor DC	 bagian wrist dan elbow merupakan motor DC yang identik.sehingga penulis hanya fokus membandingkan spesifikasi dua motor DC yaitu motor DC yang berada di bagian shoulder (\textit{main arm}) dan motor DC yang berada di bagian elbow (\textit{fore arm}). spesifikasi dua motor DC tersebut dapat dilihat pada tabel 2.2 sebagai berikut

\begin{table}[H]
	\centering
	\caption{Spesifikasi Motor DC pada robot Serpent-1}
	\resizebox{11cm}{!}{%
		\begin{tabular}{|l|l|}
			\hline
			Moments of inertia of the main arm ($J_{1}$)    							& $0.0980kgm^{2}$ 				\\ \hline
			Moments of inertia of the fore arm ($J_{2}$)    							& $0.0115 kgm^{2}$ 				\\ \hline
			Masses of the main arm	($m_{1}$)											& $1.90kg$   					\\ \hline
			Masses of the fore arm  ($m_{2}$)     										& $0.93kg$   					\\ \hline
			Motor and equivalent inertias ($J_{m}$)      								& $3.3*10^{-6}kgm^{2}$ 			\\ \hline
			Back emf constants for main arm and fore arm motor ($K_{e1}=K_{e2}$)  		& $0.047Nm/A$   				\\ \hline
			Armature resistance for main arm and fore arm motor($R_{a1}=R_{a2}$)		& $3.5\Omega$  					\\ \hline
			Armatures inductances for main and fore arm motor  ($L_{a1}=L_{a2}$) 		& $1.3mH$ 						\\ \hline
		\end{tabular}%
	}
\end{table}

\section{Kinematika Robot}
	Pada bagian motor servo, robot serpent-2 menggunakan tiga buah sensor \emph feedback yang berguna sebagai pemberi nilai posisi pada masing-masing motor servo. Sensor \emph feedback yang digunakan pada robot SCARA ini menggunakan potensiometer yang memberikan nilai analog dan kemudian diproses oleh Arduino Mega 2560. Nilai ini, nantinya untuk memproses gerak kinematika dari robot SCARA tersebut sesuai dengan posisi yang diinginkan.		
	
\section{Processing IDE}
	\begin{figure}[H]
	\centering
	\includegraphics[width=4.33cm]{gambar/logo_processing.png}
	\caption{Processing IDE}
	\end{figure}
	\emph{Processing} adalah lingkungan pemrograman sederhana yang dibuat untuk  memudahkan pengembangkan aplikasi yang berorientasi visual dengan penekanan pada animasi dan menyediakan respon balik yang instan kepada pengguna melalui interaksi didalamnnya. Para pengembang menginginkan cara untuk "membuat sketsa" ide dalam kode. Karena kemampuannya telah berkembang selama dekade terakhir, \emph{Processing} telah digunakan untuk pekerjaan tingkat produksi yang lebih maju. Awalnya dibangun sebagai ekstensi khusus domain ke Java yang ditargetkan untuk seniman dan desainer, Processing telah berevolusi menjadi desain penuh dan alat \emph{prototyping} yang digunakan untuk pekerjaan instalasi skala besar, gambar gerak, dan visualisasi data yang kompleks.

%BAB_3 LAPORAN KP

\chapter{PERANCANGAN SISTEM}

\section{Metode Perancangan sistem}
Perancangan robot serpent-2 ini diawali dengan menentukan metode yang tepat untuk mendesain dan membangun sistem secara keseluruhan meliputi perancangan elektronis,pemrograman pada Arduino Mega 2560, implementasi kinematika robot pada robot serpent-2, serta perancangan antarmuka pada \emph {processing ide}.Metode perancangan sistem meliputi diagram blok, flowchart cara kerja sistem, prinsip kerja dan perancangan tiap segmen-segmen yang dibutuhkan.
	
	\subsection{Diagram Blok Perancangan Sistem}
	Pada dasarnya, perancangan sistem untuk robot serphent-2 secara sederhana dapat dibagi menjadi tiga bagian. Ketiga perancangan ini merupakan hal yang sangat penting dan saling berkaitan.Perancangan robot serphent=2 jika digambarkan dalam diagram blok sistem dapat digambarkan seperti yang ditunjukkan dalam gambar 3.1
	\begin{figure}[H]
		\centering
		\includegraphics[width=\linewidth]{gambar/diagram_blok.jpg}
		\caption{Diagram metode perancangan sistem.}
	\end{figure}

\subsection{Flowchart Cara Kerja Sistem}
Kerja sistem, merupakan bagaimana robot serphent-2 melakukan tugasnya sesuai perintah yang dimasukkan dan kemudian dilaksanakan oleh aktuator.Robot serphent-2 memiliki kerja sistem yang tergolong ringkas yang mana didominasi oleh sistem maju tetapi juga memiliki sistem balik.  Kerja sistem dari robor serphent 2 jika dirancang dalam bentuk flowchat dapat ditunjukkan seperti dalam gambar 3.2

\begin{figure}[H]
	\centering
	\includegraphics[width=8cm	]{gambar/flowchart.png}
	\caption{Flowchat cara kerja sistem.}
\end{figure}
	
	\subsection{Perancangan Elektronis}
	Perancangan elektronis merupakan perancangan dasar pada pembuatan suatu sistem. Suatu sistem dapat bekerja secara maksimal karena terdiri dari komponen-komponen yang memiliki fungsi masing-masing. Komponen-komponen ini, disatukan kedalam sebuah \textit{Shield} \textit{Printed Circuit Board} (PCB). 
	
	\begin{enumerate}
		\item Pengendali motor DC yang digunakan adalah modul EMS 30A H-Bridge sebanyak tiga buah yang masing-masing untuk menggerakkan \textit{Shoulder, Elbow} dan perputaran \textit{Wrist}. Secara garis besar, fungsi modul pengendali motor ini adalah untuk mengendalikan arah dan kecepatan putaran motor DC sesuai instruksi kendali dari Arduino Mega 2560 pengguna.Modul akan menerima nilai yanf dikirimkan oleh Arduino Mega 2560 dan kemudian menggerakan motor servo yang sudah terhubung dengan \textit{shoulder, Elbow} dan perputaran dari \textit{Wirst}.
		
		\begin{figure}[H]
			\centering
			\includegraphics[width=5cm	]{gambar/driver_motor.jpg}
			\caption{Pengendali Motor DC EMS 30A H-Bridge}
		\end{figure}
		
		\item Potensiometer yang digunakan adalah jenis potensiometer \textit{rotary}. Potensiometer ini sebagai sensor posisi motor servo. Potensiometer terpasang pada setiap bagian motor servo sesuai dengan perputarannya dan akan memberikan keluaran berupa level tegangan yang berubah-ubah sesuai dengan posisi motor servo saat itu. Level tegangan tersebut kemudian dikirimkan kepada Arduino Mega 2560 sebagai sensor \textit{feedback} yang nantinya akan diproses untuk menyempurnakan posisi sesuai yang ditentukan.
		\begin{figure}[H]
			\centering
		%	\includegraphics[width=5cm	]{potensio.jpg}
			\caption{Mekanisme pemasangan potensiometer pada motor}
		\end{figure}
		
		\item Pengaturan pergerakan vertikal dari \textit{wirst} pada robot serphent-2 menggunakan sistem pneumatik silinder. Pada bagian buka tutup \textit{wirst} menggunakan masukan udara biasa untuk menutupnya dan membuang udara unutk membukanya. Udara tersebut didapat dari kompresor yang terhubung melalui selang dan dikontrol melalui sebuah relay yang bekerja pada tegangan 24v.
		
		\begin{figure}[H]
			\centering
		%	\includegraphics[width=5cm	]{relay.jpg}
			\caption{Relay pneumatik}
		\end{figure}
		\begin{figure}[H]
			\centering
		%	\includegraphics[width=5cm	]{relay2.jpg}
			\caption{Pneumatik Silinder}
		\end{figure}
		
		\item Relay yang bekerja pada tegangan 24v, pada Arduino Mega 2560 dikontrol melalui sinyal digital dengan bantuan rangkaian yang menggunakan TIP31A yang berfungsi untuk memutus atau membuka tegangan 24v. 
		\begin{figure}[H]
			\centering
		%	\includegraphics[width=5cm	]{relay3.jpg}
			\caption{Rangkaian skematik TIP31 sebagai \textit{switch}}
		\end{figure}
		
		\item Semua komponen-komponen yang dibutuhkan pada sistem kerja, disatukan ke dalam \textit{shield PCB} yang bertujuan agar meringkaskan serta memudahkan perangkaian elektronis. Rangkaian PCB dibuat melalui \textit{software} Eagle.
		\begin{figure}[H]
			\centering
		%	\includegraphics[width=15cm	]{skematik.pdf}
			\caption{Skematik rangkaian elektronis keseluruhan}
		\end{figure}
		
		
	\end{enumerate}
	\subsection{Perancangan Elektronis}
	
	\subsection{Pemrograman Kontroller \& Implementasi PID}
	
	\subsection{Perancangan Antar Muka \& Inplementasi \textit{Invers Kinematic}}
	

\chapter{PENGUJIAN DAN PEMBAHASAN}
Pada bab pengujian dan pembahasan ini, penulis akan melakukan pengujian sistem kendali\textit{ arm manipulator} robot SCARA berdasarkan spesifikasi sistem yang telah dijelaskan pada bab sebelumnya. Tujuan pengujian ini adalah untuk membuktikan apakah sistem yang diimplementasikan telah memenuhi spesifikasi dan rancangan yang sudah direncanakan sebelumnya. Hasil dari pengujian akan dimanfaatkan untuk menyempurnakan kinerja dari sistem dan sekaligus digunakan dalam pengembangan sistem lebih lanjut. Metode pengujian dipilih berdasarkan fungsionalitas dan beberapa parameter yang ingin diketahui dari sistem tersebut. Data yang diperoleh dari metode pengujian yang dipilih tersebut dapat memberikan informasi yang cukup dan dapat digunakan untuk penyempurnaan dan pengembangan sistem.

Metode pengujian menggunakan dua macam metode, yaitu pengujian fungsionalitas dari setiap komponen dan pengujian sistem secara keseluruhan. Pengujian fungsionalitas digunakan untuk membuktikan apakah sistem yang diimplementasikan dapat memenuhi persyaratan dari fungsi operasional yang telah dirancang dan direncanakan sebelumnya. Sedangkan pengujian sistem secara keseluruhan bertujuan untuk memperoleh beberapa parameter yang dapat menunjukkan kemampuan dan keandalan dari sistem secara keseluruhan dalam menjalankan fungsi operasionalnya. Pada \textit{sistem arm manipulator} robot SCARA dilakukan terlebih dahulu pengujian terhadap fungsional dari beberapa komponen seperti bagian \textit{DC to DC converter}, arah gerakan motor DC, \textit{feedback} potensiometer, fungsi rangkaian \textit{switching} pada \textit{valve pneumatic} dan keakuratan setiap \textit{joint} untuk bergerak sesuai sudut yang diinginkan berdasarkan kinematika balik maupun kinematika maju dengan menggunakan kontrol dari GUI.  Kemudian setelah pengujian fungsional terpenuhi maka dilakukan pengujian sistem secara keseluruhan untuk mengetahui keakuratan dan keandalan dari sistem \textit{arm manipulator} robot SCARA.

\section{Pengujian Fungsional}
Pengujian fungsional digunakan untuk menguji bagian – bagian dari sistem yang terdiri dari\textit{DC to DC converter}, arah gerakan motor DC, \textit{feedback} potensiometer, fungsi rangkaian \textit{switching} pada \textit{valve pneumatic} dan keakuratan setiap \textit{joint}, pengujian GUI Processing dan pengujian program. 

\subsection{Pengujian DC - to - DC Converter}
Pengujian DC – to – DC \textit{converter} dilakukan untuk mengetahui tegangan masukan pada Arduino Mega 2560, sensor \textit{potensiometer}, motor DC dan juga sumber tegangan untuk\textit{ valve pneumatic}. Tegangan masukan dari catu daya utama sebesar 24 Volt DC yang nantinya dibagi ke tiga buah nilai tegangan. Tabel 4.1 merupakan tegangan keluaran DC – to - DC.

\begin{table}[H]
	\centering
	\caption{Hasil Tegangan Keluaran Dari Tegangan DC-DC Converter}
	
	\begin{tabular}{|c|l|}
		\hline
		\rowcolor[HTML]{9B9B9B} 
		No & \multicolumn{1}{c|}{\cellcolor[HTML]{9B9B9B}Keterangan} \\ \hline
		1  & Robot Lengan SCARA                                      \\ \hline
		2  & Box Panel                                               \\ \hline
		3  & Arduino Mega 2560                                       \\ \hline
		4  & Personal Computer                                       \\ \hline
		5  & GUI Processing IDE                                      \\ \hline
		6  & Workspace Robot SCARA                                   \\ \hline
		7  & Kompresor                                               \\ \hline
		8  & Objek                                                   \\ \hline
	\end{tabular}
	
\end{table} 

Pada saat mengubbah besar tegangan keluaran yang dilakukan oleh Regulator \textit{Buck} LM2596 dilakukan dengan cara memutar \textit{potensiometer} yang terdapat pada Regulator \textit{Buck}. Diputar searah dengan jarum jam sesuai hingga pada teganangan yang diinginkan.
\subsection{Pengujian Motor DC}
Pengujian motor DC dilakukan untuk mengetahui apakah motor DC dalam keaadaan baik atau tidak. Pengujian dilakukan dengan memberikan tegangan kerja pada motor DC yang ada pada \textit{shoulder}, \textit{elbow}, dan juga \textit{end-effector} yang nantinya diukur arus yang dihasilkan pada masing-masing motor DC. Tabel \ref{tbl.motordc} merupakan hasil dari pengujian pada masin-masing motor DC.

\begin{table}[H]
	\centering
	\caption{Hasil Tegangan Keluaran Dari Tegangan DC-DC \textit{Converter}}
	
	\begin{tabular}{|c|l|}
		\hline
		\rowcolor[HTML]{9B9B9B} 
		\label{tbl.motordc}
		No & \multicolumn{1}{c|}{\cellcolor[HTML]{9B9B9B}Keterangan} \\ \hline
		1  & Robot Lengan SCARA                                      \\ \hline
		2  & Box Panel                                               \\ \hline
		3  & Arduino Mega 2560                                       \\ \hline
		4  & Personal Computer                                       \\ \hline
		5  & GUI Processing IDE                                      \\ \hline
		6  & Workspace Robot SCARA                                   \\ \hline
		7  & Kompresor                                               \\ \hline
		8  & Objek                                                   \\ \hline
	\end{tabular}
	
\end{table} 

Pada hasil yang ditunjukkan oleh tabel \ref{tbl.motordc} menujukkan bahwa setiap motor DC mempunyai nilai arus yang berbeda-beda. Motor DC yang terletak pada \textit{end-effcetor} merupakan motor DC yang menghasilkan arus paling besar. Hal ini disebabkan karena motor DC yang terpasang pada \textit{end-effector} dibantu dengan bantuan \textit{belt} untuk menyalurkan putaran pada\textit{ end-effector}. Dengan begitu penggunaan \textit{belt} pada motor DC ini menyebabkan beban yang dikerjakan oleh motor DC pada \textit{end-effector} menjadi lebih besar dari pada motor DC yang lain yang langsung menggerakkan pada masing-masing \textit{joint}.

\subsection{Pengujian \textit{Driver} Motor H – \textit{Bridge}}
Pengujian \textit{driver} motor H – \textit{bridge} dilakukan untuk mengetahui keberfungsian dari \textit{driver} motor apakah sesuai dengan perancangan atau tidak. Pada \textit{driver} motor juga dilakukan pengujian untuk melihat direksi dari arah pergerakan motor DC dari \textit{output} \textit{driver} ketika diberikan masukan berupa sinyal \textit{high} dan \textit{low} pada arduino. Tabel \ref{tbl.drivermotor} menunjukkan hasil dari pengujian \textit{driver} motor H-\textit{Bridge}. 
\begin{table}[H]
	\centering
	\caption{Hasil Pengujian \textit{Driver} Motor H-\textit{Bridge}}
		\label{tbl.drivermotor}
	\begin{tabular}{|c|l|}
		\hline
		\rowcolor[HTML]{9B9B9B} 
	
		No & \multicolumn{1}{c|}{\cellcolor[HTML]{9B9B9B}Keterangan} \\ \hline
		1  & Robot Lengan SCARA                                      \\ \hline
		2  & Box Panel                                               \\ \hline
		3  & Arduino Mega 2560                                       \\ \hline
		4  & Personal Computer                                       \\ \hline
		5  & GUI Processing IDE                                      \\ \hline
		6  & Workspace Robot SCARA                                   \\ \hline
		7  & Kompresor                                               \\ \hline
		8  & Objek                                                   \\ \hline
	\end{tabular}
	
\end{table} 

 Pada \textit{driver} motor H-\textit{Bridge} EMS 30A sinyal digital \textit{high} dan \textit{low} dihubungkan pada pin MEN1 dan MEN2. Dari hasil pengujian \textit{driver} motor H-\textit{Bridge} seperti yang ditunjukkan pada tabel \ref{tbl.drivermotor} terlihat bahwa ketika sinyal \textit{high} diberikan pada MEN1 dan \textit{low} diberikan MEN2 maka pergerakan motor akan berputar searah dengan arah jarum jam dan sebaliknya jika diberikan \textit{low} pada MEN1 dan \textit{high} pada MEN2 maka arah pergerakan motor akan berlawanan arah. Pada tabel \ref{tbl.drivermotor}juga terlihat bahwa nilai arus dapat dialirkan pada driver motor beragam dari 1 Ampere hingga 1.5 Ampere. Terbukti bahwa \textit{driver} motor dapat mengoperasikan driver motor dengan baik.

%BAB_5 LAPORAN KP

\chapter{PENUTUP}
\section{Kesimpulan}
Dari hasil pengujian dan pembahasan terhadap alat dan sistem “Kinematika dan Antarmuka Robot SCARA berbasis Processing IDE” yang telah dirancang dan dibuat ini, maka dapat disimpulkan sebagai berikut:
\begin{enumerate}
	\item Sistem kendali robot lengan dengan Arduino Mega 2560 dapat digunakan dengan baik dan dapat menerapkan sistem kinematika denga benar. Hasil dari nilai data yang diperoleh pada Processing IDE selanjutnya digunakan sebagai masukan untuk kinematika balik yang hasilnya dikirimkan pada Arduino Mega 2560 yang diteruskan pada motor DC.
	\item GUI yang dibuat sebagai jembatan antara user dengan hardware memiliki beberapa masukan data dan juga tampilan data berupa animasi dari robot SCARA.
	\item Implementasi dari kineatika robot SCARA dapat berjalan dengan baik meskipun terdapat beberapa data yang tidak akurat. 
	\item  Hal-hal yang menyebabkan robot lengan SCARA  tidak akurat dalam menentukan posisi end-effector diantaranya sebagai berikut:
	\subitem   Terdapat toleransi sudut pada joint shoulder dan joint elbow dan elbow yang menyebabkan $\theta_{1}, \theta_{2}$, memiliki nilai toleransi sebesar 3 derajat. 
	\subitem  Kepresisian  pembacaan nilai analog potensiometer pada feedback posisi. 
\end{enumerate}
\section{Saran}
Setelah mengambil beberapa kesimpulan dan melihat dari sistem secara keseluruhan, terdapat beberapa saran disampaikan untuk menambah mutu dan kualitas dari sistem kendali robot lengan SCARA adalah sebagai berikut: 
\begin{enumerate}
	\item Untuk menghasilkan kinematika yang baik maka toleransi besar sudut pada motor DC bisa diperkecil dan ditambahkan PID
	\item Untuk menghasilkan tampilan yang nyata maka pada sistem diberi tambahan kamera yang dapat ditampilan pada GUI
\end{enumerate}

\bibliography{IEEEabrv,references}
\addcontentsline{toc}{chapter}{DAFTAR PUSTAKA}

\end{document}