%BAB-1 Laporan KP

\chapter{PENDAHULUAN}

\section{Latar Belakang Masalah}

	Perkembangan teknologi serta ilmu pengetahuan pada masa ke masa semakin berkembang. Perkembangan ini berjalan seiring dengan penelitian-penelitian di berbagai disiplin ilmu khususnya dalam bidang instrumentasi dan kendali. Hal ini dapat dilihat dari banyaknya penggunaan sistem instrumentasi dan kendali dalam dunia industri seperti pengguanaan robot dalam menyelesaiakan pekerjaan manusia.Untuk itu perancangan robot merupakan salah satu solusi untuk memenuhi tuntutan dalam membantu kebutuhan manusia.
	
	Pemilihan robot untuk menggantikan pekerjaan manusia tidak terlepas dengan berbagai kelebihannya. Robot dapat melakukan suatu pekerjaan yang sama dan berulang tanpa merasakan lelah seperti halnya manusia. Pekerjaan ini lah yang biasa ditemukan dalam bidang industri khususnya pada bagian produksi. Robot dengan sistem lengan robot (\emph {robot arm sistem}) merupakan salah satu jenis robot yang dominan berada dalam bidang industri. 
	
	Robot lengan memiliki berbagai jenis, salah satunya adalah robot SCARA (\emph{Selective Compilance Assembly Robot Arm}). Robot SCARA dapat bergerak secara optimal dan efisien karena sebuah persamaan kinematika. Persamaan kinematika yang diguanakan adalah \textit{inverse kinematic} dengan masukan berupa titik koordinat kartesius \textit{($x_{1}$,$y_{2}$)} dan keluaran berupa nilai sudut untuk mengendalikan motor servo pada \textit{shoulder} dan \textit{elbow}.
	
	Dalam mengendalikan sebuah robot, dibutuhkan \textit{platfrom} antar muka sebagai jembatan antara \textit{user} dengan \textit{hardware}. Dalam penelitian ini program antar muka dirancang menggunakan \textit{software} \emph{Processing Ide}. \emph Software ini memiliki beberapa keunggulan yang membuatnya lebih efektif dan cukup mudah untuk digunakan sebagai \textit{platform} antar muka. Keunggulan tersebut salah satunya mudahnya sarana komunikasi terhadap \emph hardware yang digunakan. Oleh Karena itu, pada program kerja praktik ini dilakukan analisis robot SCARA berupa kinematika maju dan kinematika balik dengan perancangan antar muka berbasis \emph {Processing Ide.}\\


\section{Tujuan Penelitian}
Adapun tujuan dalam melaksanakan penelitian "Kinematika dan Antarmuka Robot SCARA Berbasis Processing IDE" adalah sebagai berikut:

	\subsection{Secara Umum}
		\begin{enumerate}
		\item Merancang \emph{ arm manipulator robot} berbasis Arduino Mega 2560
		\item Memahami dan mengimplementasikan anta muka aplikasi Processing Integrated Development Environment (IDE).
		\item Mengimplementasikan kinematika pada \emph{arm manipulator robot} SCARA.
	\end{enumerate}
	\subsection{Tujuan Khusus}
	 Untuk memenuhi salah satu syarat kelulusan dalam menempuh pendidikan Program   Diploma III Teknologi Listrik, Sekolah Vokasi, Universitas Gadjah Mada. 
	
	
\section{Batasan Penelitian}
	Pembatasan masalah diperlukan untuk mempermudah pelaksanaan penulisan laporan kerja praktik sehingga tidak menyimpang dari judul laporan. Lingkup pembatasan masalah dalam Laporan kerja praktik ini dibatasi pada :
	
	\begin{enumerate}
		
		\item Komunikasi antara LabVIEW dengan Arduino Menggunakan NI-VISA
		\item Analisis model \textit{elbow planar manipulator} Denavit-Hartenberg hanya dilakukan pada \textit{Horizontal joint}
		
	\end{enumerate}

\section{Metode Kerja Praktik}
metode yang digunakan dalam pengerjaan proyek dan penyusunan laporan kerja praktik ini dilakukan dengan metode sebagai berikut

\begin{enumerate}
	\item Metode pustaka\\
	Metode ini dilakukan dengan membaca dan mempelajari jurnal-jurnal yang memiliki topik seputar SCARA,LabVIEW, dan \textit{invers kinematic}
	
	\item Metode perancangan alat\\
	Metode ini dilakukan dengan membuat desain elektronis, perancangan kinematika robot, serta perancangan software antar muka robot.
	
	\item Metode pengujian\\
	Metode ini dilakukan dengan cara melakukan pengujian terhadap kinerja sensor, kinerja aktuator, dan kinerja sensor dan aktuator secara bersamaan.
	
\end{enumerate}

\section{Sistematika Penulisan}
Penulisan laporan kerja praktik ini dilakukan dengan mengikuti sistematika sebagai berikut :\\
\noindent
\textbf{BAB I\hspace*{0.6cm}: PENDAHULUAN}\\
\noindent
Memuat latar belakang masalah,tujuan, dan maksud,batasan masalah, metodologi penetilian, dan sistematika penulisan.\\
\noindent
\textbf{BAB II\hspace*{0.5cm}: LANDASAN TEORI}\\
\noindent
Memuat gambaran umum SCARA, \textit{invers kinematic} dengan analisis Denavit-Hartenberg, Software LabVIEW,dan Arduino Mega 2560.\\
\textbf{BAB III\hspace*{0.375cm}:  PERANCANGAN SISTEM}\\
\noindent
Memuat perancangan sistem secara umum, meliputi perancangan dari segi elektronis dan software dari robot Serpent-1.\\
\textbf{BAB IV\hspace*{0.4cm}: PENGUJIAN SISTEM}\\
\noindent
Memuat pengujian dan analisis kerja dari sistem aktuator dan sensor dari Serpent, serta pengujian sistem secara keseluruhan.\\
\textbf{BAB V\hspace*{0.6cm}: PENUTUP}\\
Memuat kesimpulan dari perancangan, pembuatan, pengujian, dan analisis kerja robot Serpent-1, serta berisi saran-saran untuk pengembangan lebih lanjut.\\